\documentclass[11pt]{article}
\usepackage[sc]{mathpazo}
\linespread{1.05}         % Palatino needs more leading (space between lines)
\usepackage[T1]{fontenc}
\usepackage{amssymb,verbatim,enumerate,amsmath,graphicx}
%\usepackage{latexsym}

\textwidth=6.5in
\textheight=9in
\topmargin=-.5in
                                                                               
\oddsidemargin=0in
\evensidemargin=0in

\def\bv{\mathbf{v}}
\def\bx{\mathbf{x}}
\def\cB{\mathcal{B}}
\def\cP{\mathcal{P}}
\def\cQ{\mathcal{Q}}
\def\QQ{\mathbb{Q}}
\def\RR{\mathbb{R}}
\def\ZZ{\mathbb{Z}}

\def\open{$[${\tt research problem}$]$ }
\def\sage{$[${\tt sage}$]$ }

\begin{document}
\setlength{\parindent}{0pt}
\setlength{\parskip}{0.4cm}

\pagestyle{empty}

\begin{center}
\Large{\bf Ehrhart Polynomials} 

\normalsize
VIII Encuentro Colombiano De Combinatoria
\end{center}

\paragraph{Day I: Appetizers}

\begin{enumerate}[(1)]
\vspace{-10pt}

\item Pick five points in $\ZZ^3$ and let $\cP$ be their convex hull (in $\RR^3$). Compute the Ehrhart polynomial of~$\cP$.

\item \sage Plot the roots of the Ehrhart polynomials of cross polytopes in
different dimensions. What's going on here?

\item Show that a sequence $f(n)$ is given by a polynomial of degree $\le d$ if and only if
    \[
        \sum_{ n \ge 0 } f(n) \, z^n \ = \ \frac{ h(z) }{ (1-z)^{ d+1 } }
    \]
    for some polynomial $h(z)$ of degree $\le d$.  Furthermore, $f(n)$ has
    degree $d$ if and only if $h(1) \ne 0$.

\item Verify (parts of) the classification picture of degree-2 Ehrhart polynomials $c_2 t^2 + c_1 t + 1$: every
half-integral point in the figure below corresponds to an Ehrhart polynomial.

\begin{center}
\includegraphics[totalheight=3.5in]{../dim2cone}
\end{center}

\item \open Give the corresponding classification picture of degree-3 Ehrhart polynomials. 
 
\end{enumerate}

% ========================================================

{\sc Matthias Beck} \hfill {\tt https://matthbeck.github.io/}

\end{document}

% ========================================================

\paragraph{II. Generating Functions}

\begin{enumerate}[(1)]
\vspace{-10pt}

\item For any polynomial $h(z)$ of degree $d$, show there exist unique polynomials $a(z)$ and $b(z)$ such that
\[
  h(z) = a(z) + z \, b(z)
  \qquad \text{ where } \qquad a(z) = z^d \, a(\tfrac 1 z) \qquad \text{ and } \qquad b(z) = z^{ d-1 } \, b(\tfrac 1 z) \, .
\]
(There are many variations of this; e.g., we could leave out the $z$ factor in front of $b(z)$.)

\item Derive inequalities for the coefficients of $h(z)$ if we know that both $a(z)$ and $b(z)$ have only nonnegative
coefficients.

\item Give an example of a polynomial $f(n)$ with (some) negative coefficients whose corresponding generating function
numerator polynomial $h(z)$ has only positive coefficients.

\item For a lattice polytope $\cP$, the numerator of the generating function is the \emph{$h^*$-polynomial} of $\cP$.
Give a non-unimodal example of an $h^*$-polynomial.

\item 
Now let
$\cP = \left\{ \bx \in [0,1]^d : \, x_1 + x_2 + \dots + x_d = k \right\}$,
for your favorite integers $1 \le k < d$. (This is the \emph{$(d, k)$-hypersimplex}.)
Prove that the $h^*$-polynomial of $\cP$ is unimodal.
(This is open.)
 
\end{enumerate}

\paragraph{IV. Betke--McMullen--Stapledon}

\begin{enumerate}[(1)]
\vspace{-10pt}

\item A linear inequality $a_0 x_0 + a_1 x_1 + \dots + a_d x_d \ge 0$ is \emph{balanced} if $a_0 + a_1 + \dots + a_d = 0$.
Find a complete set of balanced inequalities for the coefficients of every $h^*$-polynomial of degree $d$ for the first positive
integers~$d$. (This is open for $d \ge 6$.)

\item In the Ehrhart setting, the nonnegativity of $h^*(z)$ was proved in 1980, whereas the nonnegativitiy of the accompanying
polynomials $a(z)$ and $b(z)$ (which is, naturally, stronger) has been known only for 10 years. Try this same setup for your favorite (combinatorial) polynomial with nonnegative coefficients.

\end{enumerate}

\item Given a polytope $\cP \subseteq \RR^d$ with vertices $\bv_1, \bv_2, \dots, \bv_n$, randomly choose $h_1, h_2,
\dots, h_n \in \RR$, and define the new polytope $\cQ \subseteq \RR^{ d+1 }$ as the convex hull of
$(\bv_1, h_1), (\bv_2, h_2), \dots, (\bv_n, h_n)$.
The \emph{lower hull} of $\cQ$ consists of all points that are \emph{visible from below}: all points $(x_1, x_2, \dots, x_{ d+1 }) \in \cQ$ for which there is no $\epsilon > 0$ such that $(x_1, x_2, \dots, x_{ d+1 } - \epsilon) \in \cQ$.
A \emph{lower face} of $\cQ$ is a face of $\cQ$ that is in the lower hull.
Let $\pi : \RR^{ d+1 } \to \RR^d$ be the projection that forgets the last coordinate. 
Show that all lower faces of $\cQ$ are simplices, and that their projections under $\pi$ form a triangulation of~$\cP$.

\end{enumerate}


