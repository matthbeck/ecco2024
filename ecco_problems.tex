\documentclass[11pt]{article}
\usepackage[sc]{mathpazo}
\linespread{1.05}         % Palatino needs more leading (space between lines)
\usepackage[T1]{fontenc}
\usepackage{amssymb,verbatim,enumerate,amsmath,graphicx}
%\usepackage{latexsym}

\textwidth=6.5in
\textheight=9in
\topmargin=-.5in
                                                                               
\oddsidemargin=0in
\evensidemargin=0in

\def\be{\mathbf{e}}
\def\bv{\mathbf{v}}
\def\bx{\mathbf{x}}
\def\cB{\mathcal{B}}
\def\cP{\mathcal{P}}
\def\cQ{\mathcal{Q}}
\def\QQ{\mathbb{Q}}
\def\RR{\mathbb{R}}
\def\ZZ{\mathbb{Z}}
\newcommand\vol{\operatorname{vol}} 

\def\open{$[${\tt research problem}$]$ }
\def\sage{$[${\tt sage}$]$ }

\begin{document}
\setlength{\parindent}{0pt}
\setlength{\parskip}{0.4cm}

\pagestyle{empty}

\begin{center}
\Large{\bf Ehrhart Polynomials} 

\normalsize
VIII Encuentro Colombiano De Combinatoria
\end{center}

\paragraph{Day I: Appetizers}

\begin{enumerate}[(1)]
\vspace{-10pt}

\item Given integers
$a,b,c,d$, form the line segment $[(a,b), (c,d)] \subset \RR^2$ joining the points
$(a,b)$\index{line segment} and $(c,d)$. Show that the number of integer points on this line segment is $\gcd(a-c,b-d)+1$.

\item Prove that a triangle with vertices on the integer lattice has no other interior/boundary lattice points if and only if it has area $\frac 1 2$.\index{triangle}
(\emph{Hint:} You may begin by ``doubling'' the triangle to form a parallelogram.)

\item Pick four points in $\ZZ^3$ and let $\cP$ be their convex hull (in $\RR^3$). Compute the Ehrhart polynomial of~$\cP$.
(If you cannot think of a good example, consider the regular tetrahedron with vertices $\left( 0,0,0 \right) , \left( 1,1,0 \right)
, \left( 1,0,1 \right) , \left( 0,1,1 \right)$.)

\item 
Recall that the standard simplex $\Delta \in \RR^d$ is the convex hull of
the unit vectors and the origin. Verify that
\[ L_\Delta(t) = \binom{d+t} d \qquad \text{ and } \qquad L_{ \Delta^\circ }(t)
= \binom{t-1} d \, . \]
(If you'd like to amuse your colleagues, we can also write $L_{ \Delta^\circ
}(t) = (-1)^d \, \binom{d-t} d$.)

\def\Q{\mathcal Q}
\newcommand\DouPyr{\operatorname{BiPyr}} 
\def\v{{\bf v}}
\newcommand\Ehr{\operatorname{Ehr}} 

\item Given a $(d-1)$-polytope $\Q$ with vertices $\v_1, \v_2, \dots, \v_m$
such that the origin is in $\Q$, we define the bipyramid $\DouPyr(\Q)$ over
$\Q$ as the convex hull of
\[
\left( \v_1, 0 \right) , \left( \v_2, 0 \right) , \dots, \left( \v_m, 0 \right) , \left( 0, \dots, 0, 1 \right) , \ \text{ and } \left( 0, \dots, 0, -1 \right) .
\]
Show that
$ \displaystyle
\Ehr_{ \DouPyr (\Q) } (z) \ = \ \frac{ 1+z }{ 1-z } \Ehr_\Q (z) \, .
$

\item Compute the Ehrhart polynomial of the octahedron
\[
  \Diamond \ = \ \left\{ \bx \in \RR^3 : \, \left| x_1 \right| + \left| x_2 \right| + \left| x_3 \right| \leq 1 \right\}
\]
via the four different approaches outlined in the lecture:
\begin{enumerate}
  \item triangulation into 8 standard tetrahedra \& their faces
(inclusion--exclusion);
  \item disjoint triangulation into 8 standard tetrahedra;
  \item \sage interpolation;
  \item \sage generating function.
\end{enumerate}
Generalize.

\item \sage Plot the roots of the Ehrhart polynomials of cross polytopes in
different dimensions. What's going on here?

\newcommand\eulerco[2]{A \left( #1, #2 \right)} 
\item 
Define the Eulerian number
$\eulerco d k$ through\footnote{There are two slightly conflicting definitions of \emph{Eulerian numbers} in the literature: sometimes, they are defined through
$\sum_{ j \ge 0 } (j+1)^d \, z^j = \frac{ \sum_{ k=0 }^{ d } \eulerco d k z^{k}
}{ (1-z)^{ d+1 } }$ instead.}
\begin{equation*}
 \sum_{ j \ge 0 } j^d \, z^j \ = \ \frac{ \sum_{ k=0 }^{ d } \eulerco d k z^{k} }{ (1-z)^{ d+1 } } \, .
\end{equation*}
Alternatively, we may think of the polynomial $ \sum_{ k=0 }^{ d } \eulerco d k z^{k} $ is the numerator of the rational function
\[
 \left( z \frac{ d }{ dz } \right)^d \left( \frac{ 1 }{ 1-z } \right) \ = \ \underbrace{ z \frac{ d }{ dz } \cdots z \frac{ d }{ dz } }_{ d \text{ times } } \left( \frac{ 1 }{ 1-z } \right) .
\]
Prove the following:
\begin{align*}
\eulerco d k \ &= \ \eulerco d {d+1-k} ,  \\
\eulerco{ d }{ k } \ &= \ (d-k+1) \, \eulerco{ d-1 }{ k-1 } + k \, \eulerco{ d-1}{ k } , \\
\sum_{ k=0 }^{ d } \eulerco d k \ &= \ d! \, , \\
\eulerco d k \ &= \ \sum_{ j=0 }^{ k } (-1)^j \binom{ d+1 }{ j } (k-j)^d . 
\end{align*}

\item The permutahedron $\cP_d \in \RR^d$ is defined as the convex hull of
\[
  \left\{ \left( \pi(1) - 1, \, \pi(2) - 1, \dots, \, \pi(d) - 1 \right) : \, \pi \in S_d \right\} ,
\]
where $S_d$ is the set of all permutations of $\{ 1, 2, \dots, d \}$.
Show that $P_d$ is a zonotope:
\[ \cP_d = [\be_1, \be_2] + [\be_1, \be_3] + \dots + [\be_{ d-1 }, \be_d] \, ,\]
where $\be_1, \be_2, \dots, \be_d$ are the standard unit vectors.

\item Prove that $\cP_d$ tiles the hyperplane spanned by it.

\item Show that a sequence $f(n)$ is given by a polynomial of degree $\le d$ if and only if
    \[
        \sum_{ n \ge 0 } f(n) \, z^n \ = \ \frac{ h(z) }{ (1-z)^{ d+1 } }
    \]
    for some polynomial $h(z)$ of degree $\le d$.  Furthermore, $f(n)$ has
    degree $d$ if and only if $h(1) \ne 0$.

\end{enumerate}

% ========================================================

{\sc Matthias Beck} \hfill {\tt https://matthbeck.github.io/}

\newpage
\begin{center}
\Large{\bf Ehrhart Polynomials} 

\normalsize
VIII Encuentro Colombiano De Combinatoria
\end{center}

\paragraph{Day II: Generating Functions \& Complexity}

\begin{enumerate}[(1)]
\vspace{-10pt}

\item \open
Choose $d+1$ of the $2^d$ vertices of the unit $d$-cube, and let $\Delta$ be the simplex defined by their convex hull.
\begin{enumerate}[(a)]
\item Which choice of vertices maximizes $\vol \Delta$?
\item What is the maximum volume of such a $\Delta$?
\end{enumerate}

\item For any polynomial $h(z)$ of degree $d$, show there exist unique polynomials $a(z)$ and $b(z)$ such that
\[
  h(z) = a(z) + z \, b(z)
  \qquad \text{ where } \qquad a(z) = z^d \, a(\tfrac 1 z) \qquad \text{ and } \qquad b(z) = z^{ d-1 } \, b(\tfrac 1 z) \, .
\]
(There are many variations of this; e.g., we could leave out the $z$ factor in front of $b(z)$.)

\item Derive inequalities for the coefficients of $h(z)$ if we know that both $a(z)$ and $b(z)$ have only nonnegative
coefficients.

\item Verify (parts of) the classification picture of degree-2 Ehrhart polynomials $c_2 t^2 + c_1 t + 1$: every
half-integral point in the figure below corresponds to an Ehrhart polynomial.

\begin{center}
\includegraphics[totalheight=3.5in]{../dim2cone}
\end{center}

\item \open Give the corresponding classification picture of degree-3 Ehrhart polynomials. 
 
\item Give an example of a polynomial $f(n)$ with (some) negative coefficients whose corresponding generating function
numerator polynomial $h(z)$ has only positive coefficients.

\item For a lattice polytope $\cP$, the numerator of the generating function is the \emph{$h^*$-polynomial} of $\cP$.
Give a non-unimodal example of an $h^*$-polynomial.

\item \open
Now let
$\cP = \left\{ \bx \in [0,1]^d : \, x_1 + x_2 + \dots + x_d = k \right\}$,
for your favorite integers $2 \le k \le d-2$. 
(This is the \emph{$(d, k)$-hypersimplex}.)
Prove that the $h^*$-polynomial of $\cP$ is unimodal.
 
\item Given a polytope $\cP \subseteq \RR^d$ with vertices $\bv_1, \bv_2, \dots, \bv_n$, randomly choose $h_1, h_2,
\dots, h_n \in \RR$, and define the new polytope $\cQ \subseteq \RR^{ d+1 }$ as the convex hull of
$(\bv_1, h_1), (\bv_2, h_2), \dots, (\bv_n, h_n)$.
The \emph{lower hull} of $\cQ$ consists of all points that are \emph{visible from below}: all points $(x_1, x_2, \dots, x_{ d+1 }) \in \cQ$ for which there is no $\epsilon > 0$ such that $(x_1, x_2, \dots, x_{ d+1 } - \epsilon) \in \cQ$.
A \emph{lower face} of $\cQ$ is a face of $\cQ$ that is in the lower hull.
Let $\pi : \RR^{ d+1 } \to \RR^d$ be the projection that forgets the last coordinate. 
Show that all lower faces of $\cQ$ are simplices, and that their projections under $\pi$ form a triangulation of~$\cP$.

\end{enumerate}

\end{document}

% ========================================================

